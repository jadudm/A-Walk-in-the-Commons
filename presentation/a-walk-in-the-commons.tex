\documentclass{beamer} 

%%%%%%%%%%%%%%%%%%%%%%%%%%%%%%%%
% PACKAGES
% Stuff we import to make LaTeX do more.

% Gives us block comments.
\usepackage{comment}

%%%%%%%%%%%%%%%%%%%%%%%%%%%%%%%%
% THEME METADATA
% Playing with themes... what others are there?
\usetheme{Warsaw}
%\usetheme{Berlin}
% Themes can have colors as well...
% Again, what is available?
\usecolortheme{seahorse}
\usefonttheme[onlylarge]{structuresmallcapsserif}

%%%%%%%%%%%%%%%%%%%%%%%%%%%%%%%%
% PRESENTATION METADATA
\title{A Walk in the Commons} 
\author{Mel Chua, Matt Jadud} 
\date{December 2, 2011} 

\begin{document}

% Generates the title of the talk.
\titlepage

\begin{comment}
* What is open / what are open communities?
** What's open source/content?
** some projects you may have heard of (Firefox, Wikipedia, etc)
** some you may not have (Wikiotics, CivX, FreeCiv, Civicommons, Sahana, CiviCRM)
** it's not just Linux - a lot of this stuff runs on other platforms too (Windows, Mac, web-based) "no, we are not trying to get you to reinstall your computer" (but if you're interested, we're happy to help)
** the Four Freedoms (made for software)
*** Freedom / friends / ?
** creative commons (made for content)
** it's more than licensing... what's "the open source way," some characteristics of those communities (realtime transparency, etc)
\end{comment}

\begin{comment}
* What can your students do?
** One example (pick one - Fedora & your first-year class)
** Documentation
** Translation
** Gardening
** Bug fixing
** Testing
** Artwork/design
** Marketing/outreach
** Being active and vocal users (use open source in outreach/service projects - for instance, Mo & the Girl Scouts) / advocacy
** Legal/licensing work (very, *very* basic stuff)
** it's the non-programming skills that are usually in most need by these communities, because nobody knows about them / how to do them, so you can almost become domain "experts" in a project
** Creative repurposing - bringing a project into a new domain it wasn't necessarily originally designed for
\end{comment}

\begin{comment}
Researchers: they exist! (10 seconds each, no more.)
* coleman - ethics in foss communities
* krafft - innovation diffusion
* benkler - "law stuff"
* von hippel - economics
* lawler - wikiversity formation
* dennys and martin - semantic mediawiki
* davis and jabeen - legitimate peripheral participation
* government adoption paper whose author name I forgot
* Mini Case Study: CC licensed video in Zach's research
\end{comment}

\begin{comment}
* What can you do?
**  Who's there to help you?
** Who's working on this?
\end{comment}

\begin{comment}
* Why make the connection?
** Legitimate Peripheral Participation & situated learning
*** Students can engage -- the currency is desire and energy
** Community of educators
*** Local (institutional)
*** Global (distributed)
*** A community of practice as opposed to a research community
\end{comment}

\begin{comment}
* Cases
** Case study: Scientific Computing (we know this target exists)
*** Target user: a senior physics major who knows some C, and wants to either 1) contribute to a project or 2) do some research and feed it back into a community
*** Write wiki articles instead of writing a paper.
*** http://www.scipy.org/Getting_Started
*** http://www.opensourcephysics.org/
**** Extend with projects that run on the EC2 compute cloud / GPGPUs
*** Automation of experiments / Arduino
\end{comment}

\begin{comment}
* existing programmes and opportunities
** GSOC
** Creative Commons uni program
** Wikipedia Prof Program
** TOS (sorta) / POSSE? (maybe)
\end{comment}

\begin{comment}
* Challenges of doing open community stuff with students
** Some problems remaining to solve (for instance, IRC's interface sucks, community members don't always understand the "shoot self in foot" redirection need, hard to find projects, etc)
\end{comment}

% Sections will be broken into separate files soon.
\begin{frame} 
\frametitle{Outline}
\end{frame} 

\section{Background}
\begin{frame} 
\frametitle{Background}
Here is another slide. 
\end{frame}

\end{document}
